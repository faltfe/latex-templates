% ---------------------------
% Darstellung von Formeln
% ---------------------------
% math             -> inline Mathemodus
% displaymath     -> Mathemodus mit einzelner Zeile
% equation         -> Mathemodus mit einzelner Zeile und Nummerierung
% ---------------------------
\chapter{Mathematische Formeln}
    \section{Formeln}
        \SI{2,4e4}{\gram \per \mole}

    \section{Einfache Formeln}
        % ---------------------------
        % Einfacher Mathemodus inline
        % ---------------------------
        \begin{math}
            A = \dfrac{\pi}{4}\,d^2
        \end{math}
        \newline
        % ---------------------------
        % Kurzschreibweise von math
        % Beachte l2tabu
        % ftp://ftp.dante.de/tex-archive/info/german/l2tabu/l2tabu.pdf
        % ---------------------------
        \(A = \dfrac{\pi}{4}\,d^2\)

        % ---------------------------
        % Einfacher Mathemodus OHNE Nummerierung
        % equation* ist zu bevorzugen bei Verwendung von amsmath
        % ---------------------------
        Satz des Pythagoras
        \begin{displaymath}
            c^2 = a^2 + b^3
        \end{displaymath}

        % ---------------------------
        % Kurzschreibweise von displaymath
        % Beachte l2tabu
        % ftp://ftp.dante.de/tex-archive/info/german/l2tabu/l2tabu.pdf
        % ---------------------------
        Alternativ:
        \[
            a = \sqrt{c^2 - b^2}
        \]

    \section{Formeln mit Nummerierung}
        % ---------------------------
        % Mathemodus MIT Nummerierung
        % ---------------------------
        Volumenänderungsarbeit
        \begin{equation}
            W_{V12} = - \int_{V_1}^{V_2} p \cdot dV + W_{r12}
        \end{equation}

        \begin{equation}
            W_{V12} = - \int\limits_{V_1}^{V_2} p \cdot dV + W_{r12}
        \end{equation}

    \section{Geordnete Formeln}

        % ---------------------------
        % Mathemodus mit Ausrichtung
        % ---------------------------
        % Ausrichtung erfolgt mithilfe des Trenners "&"
        % Zeilenende durch \\ kennzeichnen
        % Keine Nummerierung durch \nonumber\\
        % ---------------------------
        \begin{align}
            t_u &= \sin(x) + \cos^2(x)\\
            \sigma_{out} &= \dfrac{1}{2} \exp^{b_2 - b_1}\nonumber\\
            \varrho_{12} &= \left(\dfrac{a \cdot b}{\pi}\right) \cdot \sqrt[4]{\pi}
        \end{align}